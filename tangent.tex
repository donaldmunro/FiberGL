\documentclass[fleqn]{article}
   \usepackage{amsmath,amsfonts,amssymb,amsthm,mathtools}
   \usepackage{commath}
   \usepackage[a4paper, total={205mm, 290mm}]{geometry}
   \newcommand{\bb}[1] {\mathbf{#1}}
   \newcommand{\vv}[1] {\vec{#1}}
   \newcommand{\cross} {\times}
\begin{document}
$f(x,y,z) = x^2 + y^2 + z^2 -r^2 = 0$ being the equation for a sphere of radius $r$ we
want to find some vector on the tangent plane to the sphere at $(x, y, z)$ to use
for an up vector, however there are infinetly many such vectors.
Projecting the ``standard'' up vector onto the tangent plane seems like a reasonable
compromise.

The gradient $\vv{n} = \nabla{f} = (2x, 2y, 2z)$ is normal to the sphere at
$(x, y, z)$. To find the projection $Y_p$ of the Y axis $Y = (0,1,0)$ onto the tangent to the
sphere at point $(x, y, z)$ we have

$Y_p = Y - \frac{(Y \cdot \vv{n})}{\norm{\vv{n}}^2}\vv{n} =
(-\frac{xy}{r^2}, -\frac{y^2}{r^2} + 1, -\frac{yz}{r^2})$

\end{document}